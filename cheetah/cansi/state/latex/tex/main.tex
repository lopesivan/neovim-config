\documentclass[a4paper]{article}

%encoding
%--------------------------------------
\usepackage[utf8]{inputenc}
\usepackage[T1]{fontenc}
%--------------------------------------

%Portuguese-specific commands
%--------------------------------------
\usepackage[brazilian]{babel}
%--------------------------------------

%Hyphenation rules
%--------------------------------------
\usepackage{hyphenat}
\hyphenation{mate-mática recu-perar}
%--------------------------------------

\usepackage{amsmath}
\usepackage{graphicx}

\title{Tarefa 2 Integrais múltiplas}

\author{Ivan Lopes}

\date{24 Abril, 2015}

\begin{document}
\maketitle

\LARGE Fórmulas Gerais

\newcommand{\IntegralDupla}{\iint\limits}

\begin{equation}
  \IntegralDupla_{g(Q)} \, f(x,y) \, dx\, dy =
  \IntegralDupla_Q \, f(x(u,v), y(u,v)) \left| \frac{\partial (x,y)}{\partial (u,v)}(u,v) \right|\, du\,dv
\end{equation}

\begin{equation}
  g(u,v)=(x(u,v),y(u,v))
\end{equation}

\section{Calcule $\iint\limits_B e^{-(x^2+y^2)}\,dx\,dy$, onde $B$ é o
  círculo $x^2 + y^2 \leq a^2$.}

\section{Calcule $\iint\limits_D e^{(x-y)/(x+y)}\,dx\,dy$, onde $D$ é a
  região triangular limitada pela reta $x+y=2$ e os eixos coordenados.}

\end{document}
