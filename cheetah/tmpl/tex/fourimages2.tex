\documentclass{article}
\usepackage{graphicx}
\usepackage{subcaption}
\pagestyle{empty}
\begin{document}
\begin{figure}[ht]
	\begin{subfigure}[b]{0.5\linewidth}
		\centering
		\includegraphics[width=0.75\linewidth]{example-image-a}
		\caption{Initial condition}
		\label{fig7:a}
		\vspace{4ex}
	\end{subfigure}%%
	\begin{subfigure}[b]{0.5\linewidth}
		\centering
		\includegraphics[width=0.75\linewidth]{example-image-b}
		\caption{Rupture}
		\label{fig7:b}
		\vspace{4ex}
	\end{subfigure}
	\begin{subfigure}[b]{0.5\linewidth}
		\centering
		\includegraphics[width=0.75\linewidth]{example-image-c}
		\caption{DFT, Initial condition}
		\label{fig7:c}
	\end{subfigure}%%
	\begin{subfigure}[b]{0.5\linewidth}
		\centering
		\includegraphics[width=0.75\linewidth]{example-image}
		\caption{DFT, rupture}
		\label{fig7:d}
	\end{subfigure}
	\caption{Illustration of various images}
	\label{fig7}
\end{figure}

The illustrations in figure~\ref{fig7}\ldots but in figure~\ref{fig7:d} you see\ldots

\end{document}
