\documentclass[a4paper, 12pt]{article}

\input{preamble/report.tex}

\title
{UERJ - Tópicos Especiais em Eletrônica - Lista 1}

\author{IVAN CARLOS DA SILVA LOPES \\ %autor
	      \texttt{Matricula: 202010087711} \\
	      \texttt{lopes.ivan@graduacao.uerj.br}
}

% Disciplina: IME04-00627 Introdução ao Processamento de Dados
% Turma: 2
% Professor: JOSE ROBERTO DE ALBUQUERQUE LACERDA
% AVA: SALA23IMECOMPUT

% Nome aluno: IVAN CARLOS DA SILVA LOPES
% Matricula: 202010087711
% email institucional: lopes.ivan@graduacao.uerj.br


% vim: set ts=4 sw=4 tw=64 ft=tex et :


\begin{document}

\maketitle % Print the title

\begin{enumerate}
    % 1
   \item Solução exercício 1
   \begin{enumerate}
        % a
	   \item Gráfico da função $f(x) = x^{3}-3x^{2}+1$ no intervalo $[-\frac{1}{2}) \leq x \leq 4]$
            \lstinputlisting[language=Matlab]{code/code1.m}

            \begin{figure}[h!]
                \begin{center}
                    \includegraphics[width=.85\textwidth]{code/code1.png}
                \end{center}
            \end{figure}
        % b
	   \item Determinar valores máximos e mínimos de $f(x)$ no formato $(x,f(x))$:
            \begin{enumerate}[{b}.1)]
                \item Método analítico
                    \newpage
                    \begin{figure}[h!]
                      \begin{center}
                          \includegraphics[width=.85\textwidth]{images/hands1.jpeg}
                      \end{center}
                    \end{figure}
                \item Busca exaustiva (precisão = $0,1$)

                    Encontrando os pares ordenados:
                    \lstinputlisting[language=Matlab]{code/code2.m}
                \item Busca exaustiva (precisão = $0,01$)
                    \\
                    Encontrando os pares ordenados:
                    \lstinputlisting[language=Matlab]{code/code3.m}
            \end{enumerate}
        %c
        \item Comente os resultados.
            \\
            Por se tratar de uma função bem coportada o aumento
            da precisão não altera o resultado.
   \end{enumerate}

    % 2
   \item Solução exercício 2
   \begin{enumerate}
        % a
		\item Gráfico da função $f(x) = -3x^{4}+16x^{3}-18x^{2}$ no intervalo $[-1 \leq x \leq 4]$

            \lstinputlisting[language=Matlab]{code/code4.m}

            \begin{figure}[h!]
                \begin{center}
                    \includegraphics[width=.85\textwidth]{code/code4.png}
                \end{center}
            \end{figure}
        % b
		\item Determinar valores máximos e mínimos de $f(x)$ no formato $(x,f(x))$:
			\begin{enumerate}[{b}.1)]
				\item Método analítico
                    \includepdf[pages=1]{images/hand2}
                    \includepdf[pages=2]{images/hand2}
				\item Busca exaustiva (precisão = $0,1$)
                    \lstinputlisting[language=Matlab]{code/code5.m}
            \end{enumerate}
        %c
        \item Comente os resultados.
            \\
            Da mesma forma que a primeira o aumento da precisão
            não altera de forma siguinificativa a resposta.
   \end{enumerate}

    % 3
   \item Solução exercício 3
   \begin{enumerate}
        % a
		\item Gráfico da função $f(x) = x sen(10\pi x) + 1$ no intervalo $[-1 \leq x \leq 2]$
            \\
            Definindo a função:
            \lstinputlisting[language=Matlab]{code/f.m}

            Plotando a função:
            \lstinputlisting[language=Matlab]{code/code6.m}

            \begin{figure}[h!]
                \begin{center}
                    \includegraphics[width=.85\textwidth]{code/code6.png}
                \end{center}
            \end{figure}
        % b
		\item Determinar valores máximos e mínimos de $f(x)$ no formato $(x,f(x))$:
			\begin{enumerate}[{b}.1)]
				\item Método analítico
				\item Busca exaustiva (precisão = $0,1$) e (precisão = $0,01$)
                    \\
                    precisão = $0,1$:
                    \lstinputlisting[language=Matlab]{code/code7.m}

                    \newpage
                    precisão = $0,01$:
                    \lstinputlisting[language=Matlab]{code/code8.m}
            \end{enumerate}

            A função possui um valor elevado de oscilações,
            sendo assim, o aumento da precisão é muito
            siguinificativo sobre a resposta.
   \end{enumerate}

    % 4
   \item Solução exercício 4
	\\
		função $f(x,y) = x sen(4 x) + 1,1 y sen(2 y)$ no intervalo $[0 \leq x \leq 10]$, $[0 \leq y \leq 10]$
   \begin{enumerate}
        % a
		\item Busca exaustiva (precisão = $0,1$)
            \\
            precisão = $0,1$:
            \lstinputlisting[language=Matlab]{code/code9.m}
        % b
		\item Busca exaustiva (precisão = $0,01$)
            \\
            precisão = $0,01$:
            \lstinputlisting[language=Matlab]{code/code10.m}
        %c
		\item Busca exaustiva (precisão = $0,001$)
            \\
            precisão = $0,001$:
            \lstinputlisting[language=Matlab]{code/code11.m}
        %d
        \item Comente os resultados.
            \\
            Como a faixa de valores é muito extensa, mudamos
            nosso algorítmo para não acumular valores no formato
            vetorial, processando de forma direta as comparações
            do loop.
   \end{enumerate}

%%%%%%%%%%%%%%%%%%%%%%%%%%%%%%%%%%%%%%%%%%%%%%%%%%%%%%%%%%%%%%%%%%%%%%%%%%%%%%
\end{enumerate}

\end{document}

% vim: set ts=4 sw=4 tw=64 ft=tex et :
